% for section-numbered lemmas etc., use "numberwithinsect"
\documentclass[a4paper,english,numberwithinsect]{eurocg18}

% the recommended bibstyle
\bibliographystyle{plainurl}

%------------------------------------------------------------------- 
%if unwanted, comment out or use option "draft"
\usepackage{todonotes}
\usepackage{microtype}
\usepackage[noend, linesnumbered]{algorithm2e}
\usepackage{amsmath}
\usepackage{color}
\usepackage{cite}
\usepackage{pifont}
\usepackage{enumerate}

% Line numbers are helpful for refereeing
\usepackage[mathlines]{lineno}
\newcommand*\patchAmsMathEnvironmentForLineno[1]{%
\expandafter\let\csname old#1\expandafter\endcsname\csname #1\endcsname
\expandafter\let\csname oldend#1\expandafter\endcsname\csname end#1\endcsname
\renewenvironment{#1}%
     {\linenomath\csname old#1\endcsname}%
     {\csname oldend#1\endcsname\endlinenomath}}%
\newcommand*\patchBothAmsMathEnvironmentsForLineno[1]{%
  \patchAmsMathEnvironmentForLineno{#1}%
  \patchAmsMathEnvironmentForLineno{#1*}}%
\AtBeginDocument{%
\patchBothAmsMathEnvironmentsForLineno{equation}%
\patchBothAmsMathEnvironmentsForLineno{align}%
\patchBothAmsMathEnvironmentsForLineno{flalign}%
\patchBothAmsMathEnvironmentsForLineno{alignat}%
\patchBothAmsMathEnvironmentsForLineno{gather}%
\patchBothAmsMathEnvironmentsForLineno{multline}%
}
\linenumbers

%helpful if your graphic files are in another directory
%\graphicspath{{./graphics/}}

% Author macros::begin %%%%%%%%%%%%%%%%%%%%%%%%%%%%%%%%%%%%%%%%%%%%%%%%
\newcommand\polylog{{\rm polylog}}

% Author metadata::begin %%%%%%%%%%%%%%%%%%%%%%%%%%%%%%%%%%%%%%%%%%%%%%%%
\title{Maximizing Ink in Symmetric Partial Edge Drawings of $k$-plane Graphs}
%{Maximal Symmetric Partial Edge Drawings With Degree $ > 2$}
%optional, in case that the title is too long; 
%the running title should fit into the top page column
\titlerunning{MaxSPEDs with degree $ > 2$}

%% Please provide for each author the \author and \affil macro, 
%even when authors have the same affiliation, i.e. for each 
%author there needs to be the  \author and \affil macros
\author[1]{Michael H\"oller}
\author[2]{Fabian Klute}
\author[3]{Soeren Nickel}
\author[4]{Martin N\"ollenburg}
\author[5]{Birgit Schreiber}

\affil[1]{TU Vienna \texttt{}}
\affil[2]{Algorithms and Complexity Group TU Vienna \texttt{fklute@ac.tuwien.ac.at}}
\affil[3]{TU Vienna \texttt{}}
\affil[4]{Algorithms and Complexity Group TU Vienna \texttt{noellenburg@ac.tuwien.ac.at}}
\affil[5]{TU Vienna \texttt{}}
%mandatory. First: Use abbreviated first/middle names. 
%Second (only in severe cases): Use first author plus 'et. al.'
\authorrunning{F. Klute and M. N\"ollenburg} 

% Author macros::end %%%%%%%%%%%%%%%%%%%%%%%%%%%%%%%%%%%%%%%%%%%%%%%%%

\newcommand{\martin}[1]{\todo[inline,color=blue!40]{MN: #1}}
\newcommand{\fabian}[1]{\todo[inline,color=pink!40]{FK: #1}}
\newcommand{\birgit}[1]{\todo[inline,color=red!40]{BS: #1}}
\newcommand{\michael}[1]{\todo[inline,color=green!40]{MH: #1}}
\newcommand{\soeren}[1]{\todo[inline,color=orange!40]{SN: #1}}

\begin{document}

\maketitle

\begin{abstract}

\end{abstract}

\section{Introduction}

Martin

\begin{itemize}
	\item motivation
	\item related work
	\item contribution
\end{itemize}

\section{Preliminaries}

Fabian

\begin{itemize}
	\item basic definitions and notation
	\item segment intersection graph
	\item alle planaren Graphen können auftreten (insbes. Bäume und Kakteen)
\end{itemize}

Input: $k$-plane graph drawing $\Gamma$ of a graph $G$ with edge set $S={s_1, \dots, s_m}$ (set of segments)

Intersection graph $C=(V,E)$ with $V=S$ and edges if two segments intersect 

Consider maximum degree of $C$ -- here mostly 3.


\section{Hardness}
\fabian{Use the reduction from Till Bruckdorfers Dissertation \cite{bruckdorferschematics}. The Clauses can be modelled as triangles, the Variables as Circles with every second segment connected to a path. Path should have even length, segments are of length four.}

Soeren?


\section{Two Polynomial Cases}

\begin{itemize}
	\item dynamic programming
	\item table entries $T_i(s)$ for $i = 1, \dots, \deg(s)$
	\item $i$ corresponds to increasing stub lengths
	\item stub lengths $l_i(s)$ for $i = 1, \dots, \deg(s)$ (not the pair of stubs)
\end{itemize}

\subsection{Trees}
\fabian{Algorithm running in $O(kn)$ where $ k $ is the max degree of the tree + intersection graph bauen.}

Birgit

\begin{itemize}
	\item segment $s$ with $d$ children $u_1, \dots, u_{d}$ and parent $p(s) = p$.
	\item Assume children are sorted by their intersection points on $s$, i.e., $u_1$ produces stub length $l_1$ etc.
	\item for stub length relating to parent use offset from next longer stub length
\end{itemize}

\subsection{Cacti (of Degree Three)}
\fabian{Algorithm running in $ O(kn) $ + intersection graph bauen.}

\begin{itemize}
	\item seems to work for all cactus graphs
\end{itemize}

\section{Conclusion}


\bibliography{paper}

\end{document}
